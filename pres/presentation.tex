\documentclass[usenames,dvipsnames,aspectratio=169]{beamer}

\usepackage[utf8]{inputenc}
\usepackage[T1]{fontenc}
\usepackage{emerald}
\usepackage{graphicx}
\usepackage{float}
\usepackage[backend=biber,style=ieee]{biblatex}
\usepackage{minted}
\usepackage{amsmath}
\usepackage{amssymb}
\usepackage{fontspec}

%%%%%%%%%%%%%%%%%%%%%%
% SETUP BEAMER THEME %
%%%%%%%%%%%%%%%%%%%%%%
\setbeamercolor{headline}{fg=white}
\setbeamercolor{institute}{fg=white}
\setbeamercolor{date}{fg=white}
\setbeamercolor{author}{fg=white}
\setbeamercolor{author}{fg=white}
\setbeamercolor{section in toc}{fg=white}
\setbeamercolor{subsection in toc}{fg=white}
\setbeamercolor{subsubsection in toc}{fg=white}
\setbeamercolor{frametitle}{fg=white}
\setbeamercolor{structure}{fg=white}
\setbeamercolor{item}{fg=white}
\setbeamercolor{itemize item}{fg=white}
\setbeamercolor{item projected}{fg=white}
\setbeamercolor{alerted text}{fg=white}
\setbeamercolor{normal text}{fg=white}
\setbeamercolor{block body}{fg=black,bg=white}

\setbeamerfont{frametitle}{size=\huge, series=\ECFAugie}
\setbeamerfont{framesubtitle}{size=\large, series=\ECFAugie}
\setbeamerfont{title}{size=\huge, series=\ECFAugie}
\setbeamerfont{author}{size=\large, series=\ECFAugie}
\setbeamerfont{date}{size=\large, series=\ECFAugie}
\setbeamerfont{institute}{size=\large, series=\ECFAugie}
\setbeamerfont{caption}{series=\ECFAugie}
\setbeamerfont{section in toc}{size=\small, series=\ECFAugie}
\setbeamerfont{subsection in toc}{size=\small, series=\ECFAugie}
% \setmainfont{\ECFAugie}

\setbeamercolor{background canvas}{bg=black}
\setbeamertemplate{background}
{\includegraphics[width=\paperwidth,height=\paperheight]{background.jpg}}

\setbeamertemplate{itemize item}{\usebeamercolor[fg]{item}\small\ECFAugie{-}\usebeamercolor[fg]{itemize item}}

\setbeamertemplate{frametitle}{
    \begin{beamercolorbox}[center]{frametitle}
        \vskip17pt
        \usebeamerfont{frametitle}
        \insertframetitle \\
        \vspace{0.5em}
        \usebeamerfont{framesubtitle}\insertframesubtitle
    \end{beamercolorbox}
}

\setbeamertemplate{section page}
{
    \begingroup
    \begin{beamercolorbox}[sep=12pt,center]{section title}
    \usebeamerfont{section title}\insertsection\par
    \end{beamercolorbox}
    \endgroup
}

\setbeamertemplate{subsection page}
{
    \begingroup
    \begin{beamercolorbox}[sep=12pt,center]{subsection title}
    \usebeamerfont{subsection title}\insertsubsection\par
    \end{beamercolorbox}
    \endgroup
}

\setbeamertemplate{bibliography item}[text]

%% remove navigation symbols
\setbeamertemplate{navigation symbols}{}

%%%%%%%%%%%%%
% PDF SETUP %
%%%%%%%%%%%%%
\hypersetup{pdfstartview={Fit}}

%%%%%%%%%%%%%%
% REFERENCES %
%%%%%%%%%%%%%%
\addbibresource{references.bib}
\renewcommand{\bibfont}{\footnotesize}

%%%%%%%%%%%%
% LISTINGS %
%%%%%%%%%%%%
\usemintedstyle{tango}

\usepackage{newunicodechar}
\newfontfamily{\freeserif}{DejaVu Sans}
\newunicodechar{ℕ}{\freeserif{ℕ}}
\newunicodechar{ₐ}{\freeserif{ₐ}}
\newunicodechar{₁}{\freeserif{₁}}
\newunicodechar{∈}{\freeserif{∈}}
\newunicodechar{𝓞}{\ensuremath{\mathcal{O}}}
\newunicodechar{∉}{\freeserif{∉}}
\newunicodechar{Π}{\freeserif{Π}}
\newunicodechar{⦃}{\freeserif{⦃}}
\newunicodechar{⦄}{\freeserif{⦄}}
\newunicodechar{ᵇ}{\freeserif{ᵇ}}


%%%%%%%%%%%%%%%%%
% MATH COMMANDS %
%%%%%%%%%%%%%%%%%
\newcommand{\N}{\mathbb{N}}

%%%%%%%%%%%%%%%%%%%%
% TITLE PAGE SETUP %
%%%%%%%%%%%%%%%%%%%%
\author{Carson Storm}
\date{November 13 2020}
\title{Formal Verification in Rust}
\institute[]{University of Utah}

\begin{document}

\maketitle

%%%%%%%%%%
% SLIDES %
%%%%%%%%%%

\begin{frame}{Outline}
    \tableofcontents
\end{frame}

\section{Why Rust?}
\begin{frame}{Why Rust?}
    \begin{itemize}
        \item a low-level language that offers zero cost abstraction
        \item a multi-paradigm language
        \item guarantees memory safety at compile time
    \end{itemize}
\end{frame}

\subsection{Rust memory safety}
\begin{frame}[fragile]{Rust memory safety (moves)}
    \begin{block}{}
        \begin{minted}[autogobble]{rust}
            fn first<S,T>((s,_):(S,T)) -> S {
                s
            }
            ...
            
            let p: (S,T) = ...;
            let s = first(p);
            // let t = second(p); // error[E0382]: use of moved value: `p`
        \end{minted}   
    \end{block}
\end{frame}

\begin{frame}[fragile]{Rust memory safety (borrows I)}
    \begin{block}{}
        \begin{minted}[autogobble]{rust}
            fn first<S,T>(&(ref s,_):&(S,T)) -> &S {
                s
            }
            ...
            
            let p: (S,T) = ...;
            let s = first(&p);
            let t = second(&p);
        \end{minted}
    \end{block}
\end{frame}

\begin{frame}[fragile]{Rust memory safety (borrows II)}
    \begin{block}{}
        \begin{minted}[autogobble]{rust}
            fn first<S,T>(&(ref s,_):&(S,T)) -> &S {
                s
            }
            ...
            
            let p: (S,T) = ...;
            let s = first(&p);      // immutable borrow occurs here
            let t = second(&mut p); // mutable borrow occurs here

            // immutable borrow ends here
            // will result in compiler error
        \end{minted}
    \end{block}
\end{frame}

\begin{frame}[fragile]{Rust memory safety (lifetimes I)}
    \begin{block}{}
        \begin{minted}[autogobble]{rust}
            fn first<S,T>(&(ref s,_):&(S,T)) -> &S {
                s
            }
            ...
            let s = {
                let p: (S,T) = ...;
    
                return first(&p);
                // p is freed here, but we are trying to return a reference to it
                // this will result in a compilation error
            }
        \end{minted}
    \end{block}
\end{frame}

\begin{frame}[fragile]{Rust memory safety (lifetimes II)}
    \begin{block}{}
        \begin{minted}[autogobble]{rust}
            fn first<'a, S,T>(&(ref s,_): &'a (S,T)) -> &'a S {
                s
            }
            ...
            let p: (S,T) + 'a = ...;
            let s: &'a S = first(&p);
        \end{minted}
    \end{block}
\end{frame}

\begin{frame}[fragile]{Rust memory safety}
    \begin{itemize}
        \item All references are guaranteed to point to valid memory
        \item Alias is allowed, but only for non mutable references
            \begin{block}{}
                \begin{minted}[autogobble]{rust}
                    fn compute(input: &u32, output: &mut u32) {
                        if *input > 10 { *output = 1; }
                        if *input > 5 { *output *= 2;}
                    }
                    ...
                    compute(&x, &mut x); // this will result in a compiler error
                \end{minted}
            \end{block}
        \item Eliminates data races
    \end{itemize}
\end{frame}

\subsection{Rust vs. C}
\begin{frame}[fragile]{Rust vs. C}
    \begin{block}{}
        \begin{minted}[autogobble]{c}
            void client(list * a, list *b) {
                int old_len = b->len;
                append(a, 100);
                assert(b->len == old_len);
            }
        \end{minted}
    \end{block}
    \begin{itemize}
        \item Could have memory errors in \mintinline[bgcolor=white]{c}{b->len}.
        \item Could have aliasing (eg. if \mintinline[bgcolor=white]{c}{a = b}).
        \item Could have data races if another thread mutates \mintinline[bgcolor=white]{c}{a} or \mintinline[bgcolor=white]{c}{b}.
    \end{itemize}
\end{frame}

\begin{frame}[fragile]{Rust vs. C}
    \begin{block}{}
        \begin{minted}[autogobble]{rust}
            fn client(a: &mut List, b: &mut List) {
                let old_len = b.len();
                append(a, 100);
                assert!(b->len == old_len);
            }
        \end{minted}
    \end{block}
    \begin{itemize}
        \item Memory is always valid so \mintinline[bgcolor=white]{c}{b.len()} will not result in a memory error.
        \item Mutable references cannot alias, so \mintinline[bgcolor=white]{c}{a} and \mintinline[bgcolor=white]{c}{b} are disjoint.
        \item Only one mutable reference is allowed at a time, so another thread cannot mutate \mintinline[bgcolor=white]{c}{a} or \mintinline[bgcolor=white]{c}{b}.
    \end{itemize}
\end{frame}

\section{Formal Verification in Rust with Lean}
\begin{frame}\sectionpage\end{frame}

\subsection{What is Lean?}
\begin{frame}[fragile]{What is Lean?}
    \begin{itemize}
        \item Lean is a theorem prover and programming language developed at 
            Microsoft Research
        \item Provides some automatic theorem proving, but focuses on the 
            verification of theorem.
        \item Has been used to formalize abstract algebra, group theory, number theory,
            topology, computability, etc.
        \item Based on Dependent Type Theory
    \end{itemize}

    \begin{block}{}
        \begin{minted}[breaklines, autogobble, bgcolor=white]{lean}
            theorem halting_problem (n) : ¬ computable_pred (λ c, (eval c n).dom)
            | h := rice {f | (f n).dom} h nat.partrec.zero nat.partrec.none trivial
        \end{minted}
    \end{block}
\end{frame}

\subsubsection{Type Theory}
\begin{frame}{Type Theory}{What is Type Theory?}
    \begin{itemize}
        \item A logical system that can server as a replacement for set theory
        \item Judgments are made through type checking. A proof of a theorem is
            considered valid if it has the same type as the theorem.
        \item Does not depend on implementation details. 
    \end{itemize}
\end{frame}

\begin{frame}{Type Theory}{Constructive vs. Classical Logic}
    \begin{itemize}
        \item Constructive logic must provide an example for proposition of the
            form
            $$
                \exists x \in \N, x > 1
            $$
            where as, in classical logic, an example is not always known.

        \item Does not allow for statements such as $p \land \lnot p$ or $\lnot \lnot p \implies p$.
            \begin{itemize}
                \item This allows for type theory to handle undecidable problems
                \item It's not always true that $p$ is either true of false.
            \end{itemize}
    \end{itemize}

\end{frame}

\subsubsection{Lean Demo}
\begin{frame}
    \begingroup
    \begin{beamercolorbox}[sep=12pt,center]{section title}
    \usebeamerfont{frametitle}{Lean Demo}
    \end{beamercolorbox}
    \endgroup
\end{frame}

\subsection{Functional Purification}
\begin{frame}[fragile]{Functional Purification}
    Since mutability is localized in Rust we perform functional rewrites
    \begin{columns}
        \column{.5\textwidth}
        \begin{block}{}
            \begin{minted}[autogobble]{rust}
                p.x += 1
            \end{minted}
        \end{block}

        \begin{block}{}
            \begin{minted}[autogobble]{rust}
                // where f(& mut Point) -> A
                let x = f(&mut p) 
            \end{minted}
        \end{block}
        
        \column{.5\textwidth}
        \begin{block}{}
            \begin{minted}[autogobble]{rust}
                let p = Point {x = p.x + 1, ..p}
            \end{minted}
        \end{block}

        \begin{block}{}
            \begin{minted}[autogobble]{rust}
                // where f(Point) -> (A, Point)
                let (x,p) = f(p) 
            \end{minted}
        \end{block}
    \end{columns}
\end{frame}

\begin{frame}[fragile]{Specifications}
    \begin{block}{}
        \begin{minted}[autogobble, breaklines]{lean}
            theorem core.slice.binary_search.sem : ∀ {T : Type} [Ord' T]
                (self : list T) (needle : T), sorted le self →
                option.any binary_search_res (binary_search self needle)
        \end{minted}
    \end{block}
\end{frame}

\begin{frame}[fragile]{Specifications}
    \begin{block}{}
        \begin{minted}[autogobble,breaklines]{lean}
            inductive binary_search_res : Result usize usize → Prop :=
            -- if the value is found then Ok is returned, containing the index of the matching element
            | found     : Πi, nth self i = some needle → binary_search_res (Result.Ok i)
            -- if the value is not found then Err is returned, containing the index where a matching
            -- element could be inserted while maintaining sorted order.
            | not_found : Πi, needle ∉ self → sorted le (insert_at self i needle) →
                binary_search_res (Result.Err i)
        \end{minted}
    \end{block}
\end{frame}

\begin{frame}[fragile]{Translation}
    \begin{columns}
        \column{.5\textwidth}
        \begin{block}{}
            \begin{minted}[autogobble]{rust}
                fn f(i: i32) -> i32 {
                    if i == 42 {
                        1
                    }
                    else {
                        0
                    }
                }
            \end{minted}
        \end{block}

        \column{.5\textwidth}
        \begin{block}{}
            \begin{minted}[autogobble, breaklines]{lean}
                definition test.f (iₐ : i32) : sem (i32) :=
                let' «i$2» ← iₐ;
                let' t4 ← «i$2»;
                let' t3 ← t4 =ᵇ (42 : int);
                if t3 = bool.tt then
                let' ret ← (1 : int);
                return (ret)
                else
                let' ret ← (1 : int);
                return (ret)
            \end{minted}
        \end{block}
    \end{columns}
\end{frame}

\begin{frame}[fragile]{Translation}
    \begin{columns}
        \column{.5\textwidth}
        \begin{block}{}
            \begin{minted}[autogobble]{rust}
                struct Point {x: i32, y: i32}

                fn main() {
                    let p = Point {x: 10, y: 11};
                    let px: i32 = p.x;
                }
            \end{minted}
        \end{block}

        \column{.5\textwidth}
        \begin{block}{}
            \begin{minted}[autogobble, breaklines]{lean}
                structure test.Point := mk {} ::
                (x : i32)
                (y : i32)

                definition test.main : sem (unit) :=
                let' «p$1» ← test.Point.mk (10 : int) (11 : int);
                let' t3 ← (test.Point.x «p$1»);
                let' «px$2» ← t3;
                let' ret ← ⋆;
                return (⋆)
            \end{minted}
        \end{block}
    \end{columns}
\end{frame}

\subsubsection{Electrolysis Demo}
\begin{frame}
    \begingroup
    \begin{beamercolorbox}[sep=12pt,center]{section title}
    \usebeamerfont{frametitle}{Electrolysis Demo}
    \end{beamercolorbox}
    \endgroup
\end{frame}

\section{Formal Verification in Rust with Prusti}
\begin{frame}\sectionpage\end{frame}

\subsection{How is Prusti different?}
\begin{frame}[fragile]{How is Prusti different?}
    \begin{itemize}
        \item Relies on a SMT solver to verify program
        \item Embeds specification directly in source
    \end{itemize}
    \begin{block}{}
        \begin{minted}[autogobble]{rust}
            #[ensures(result >= a && result >= b)]
            #[ensures(result == a || result == b)]
            fn max(a: i32, b: i32) -> i32 {
                if a < b {
                    b
                } else {
                    a
                }
            }
        \end{minted}
    \end{block}
\end{frame}

\subsection{Prusti Design}
\begin{frame}{Prusti Design}
    \begin{block}{}
        \centering\includegraphics[width=\textwidth]{prusti_workflow.png}
    \end{block}
\end{frame}

\subsection{What is Viper?}
\begin{frame}[fragile]{What is Viper?}
    \begin{itemize}
        \item A language independent verification tool built on top of Z3
        \item Prusti transpiles the rust source to viper statements
    \end{itemize}
\end{frame}

\begin{frame}[fragile]{Viper Translation}
    \begin{columns}
        \column{.5\textwidth}
        \begin{block}{}
            \begin{minted}[autogobble]{rust}
                struct List {
                    val: i32;
                    next: Option<Box<List>>
                }
            \end{minted}
        \end{block}
        \begin{block}{}
            \begin{minted}[autogobble]{rust}
                fn client(a:&mut List, b:&mut List)
            \end{minted}
        \end{block}

        \column{.5\textwidth}
        \begin{block}{}
            \begin{minted}[autogobble]{text}
                predicate List(self: Ref) {
                    acc(self.val) *
                    acc(self.next) *
                    i32(self.val) *
                    OptionBoxList(self.next)
                }
            \end{minted}
        \end{block}
        \begin{block}{}
            \begin{minted}[autogobble]{text}
                method client(a : Ref, b: Ref)
                    requires List(a) * List(b)
                    ensures List(a) * List(b)
            \end{minted}
        \end{block}
    \end{columns}
\end{frame}

\subsection{Prusti Demo}
\begin{frame}
    \begingroup
    \begin{beamercolorbox}[sep=12pt,center]{section title}
    \usebeamerfont{frametitle}{Prusti Demo}
    \end{beamercolorbox}
    \endgroup
\end{frame}

\section{Further Reading}
\begin{frame}[allowframebreaks]
    \frametitle{Further Reading}
    \nocite{*}
    \printbibliography
\end{frame}
    
\end{document}